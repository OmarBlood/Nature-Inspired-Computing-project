\documentclass[xcolor={usenames,dvipsnames,svgnames}]{beamer}

\usetheme{mun}
\usepackage{array}
\usepackage{biblatex}
\bibliography{\jobname}

\title{GAs to Solve the TSP}
\subtitle{Computer Science 3201}
\author{Jacob House // Nabil Miri \\ Omar Mohammed // Hassan El-Khatib}
\date{Fall 2018}

\begin{document}
\begin{frame}[plain]
\titlepage
\end{frame}

\startheads

\section*{Outline}
\begin{frame}
	\tableofcontents
\end{frame}

\section{The Team}
\begin{frame}
\def\arraystretch{2}%  1 is the default, change whatever you need
\begin{tabular}{>{\color{structure}\bfseries}ll}
	Omar Mohamed & \parbox[t]{.75\linewidth}{Project management \\ Programmer} \\
	Nabil Miri & \parbox[t]{.75\linewidth}{Algorithm implementation \\ Debugging} \\
	Jacob House & \parbox[t]{.75\linewidth}{Technical management \\ Code quality control} \\
	Hassan El-Khatib & Programmer
\end{tabular}
\end{frame}

\section{Our Approach}
\subsection{Population Size}
\begin{frame}
\begin{itemize}[<+->]
	\item For a route with $n$ cities, we have
	\begin{equation*}
	R := n \cdot (n-1) \cdots 3 \cdot 2 \cdot 1 = n!
	\end{equation*}
	possible routes that cover all cities
	\item As $n$ grows, so does $R$
	\item Population size $P$ should also grow with $n$
	\item We define $P := 2n$
\end{itemize}
\end{frame}


\section{References}
\begin{frame}[allowframebreaks]
\nocite{*}
\printbibliography
\end{frame}
\end{document}