%% BREAK LINES EVERY 80 CHARACTERS TO HELP GIT WITH MERGING
\subsection{The NumPy Module}
Before any benchmarking was done, an effort was made to use the 
NumPy \cite{numpy} module's interfaces as much as possible over raw 
Python data types. For example, when possible, |numpy.ndarray| was 
chosen over Python's |list| data structure. The SciPy\footnote{NumPy is
part of the SciPy Stack.} documentation states:
\begin{quote}
	 \ldots the fact that [Python |list|s] can contain objects of differing 
	 types mean that Python must store type information for every 
	 element, and must execute type dispatching code when operating 
	 on each element. This also means that very few list operations can 
	 be carried out by efficient C loops --- each iteration would require 
	 type checks and other Python API bookkeeping. \cite{scipy_docs}
\end{quote}
As the problem being solved requires extensive accessing and mutating
of arrays containing a single data type (\ie real numbers), NumPy 
provided a very easily-implemented speed boost.

\subsection{Cheaply Computing Distance}
Suppose that the distance between cities $C_j$ and $C_{j+1}$ is not less
 than $1$ and the same applies for cities $C_k$ and $C_{k+1}$. That is, 
\begin{align*}
\textstyle 1 &\leqslant \left\lVert \overrightarrow{C_j \, C_{j+1}} \right\rVert \\
&= \sqrt{(x_{C_{j+1}} - x_{C_j})^2 + (y_{C_{j+1}} - y_{C_j})^2}
\end{align*}
and
\begin{align*}
\textstyle 1 &\leqslant \left\lVert \overrightarrow{C_k\, C_{k+1}} \right\rVert \\
&= \sqrt{(x_{C_{k+1}} - x_{C_k})^2 + (y_{C_{k+1}} - y_{C_k})^2}
\end{align*}
Then 
$\left\lVert \overrightarrow{C_j\, C_{j+1}} \right\rVert < \left\lVert \overrightarrow{C_k\, C_{k+1}} \right\rVert$ 
if, and only if, 
$\left\lVert \overrightarrow{C_j\, C_{j+1}} \right\rVert^2 < \left\lVert \overrightarrow{C_j\, C_{j+1}} \right\rVert^2$.
Hence, by adding an |assert| statement to our code to ensure that 
\begin{equation*}
(x_{C_{j+1}} - x_{C_j})^2 + (y_{C_{j+1}} - y_{C_j})^2 \geqslant 1,
\end{equation*}
we can omit the expensive square root operation $\frac{1}{2}(n^2 + n)$ 
times\footnote{If distances are computed only once (see
sub-section~\ref{ssec:precomputing}). Otherwise the number of expensive
calculations may be higher.} and still maintain accurate distance 
comparisons and fitness scores. 

After the final iteration of the GA, the true distance must still be calculated
using \eqref{eq:fit1}.

\subsection{Maximizing Hardware Utilization}
At the time of writing, nearly all modern computers utilize multiple processor
cores (or in some cases multiple processors, each with multiple cores) to 
accomplish tasks\footnote{Specifications of several of Memorial's LabNet 
computers used for computation are included in Appendix~\ref{sec:comp-specs}.}.
Natively, the Python language uses only {\em one} core, leaving the rest 
underutilized. 

Immediately, multi-threaded processing using Python's 
|multiprocessing.dummy| module was investigated to take full 
advantage of the computer's hardware. Multithreading was chosen over
multiprocessing as this does not impose the requirement that functions 
be serializable (referred to by the term {\em pickleable} in Python). This 
allows local helper functions and lambda expressions to be used.
This resulted in {\em worse} performance because, as QuantStart explains,
\begin{quote}
	...the Python interpreter is not thread safe. This means that there is a
	globally enforced lock when trying to safely access Python objects from 
	within threads. At any one time only a single thread can acquire a lock 
	for a Python object or C API. The interpreter will reacquire this lock for 
	every 100 bytecodes of Python instructions and around (potentially) 
	blocking I/O operations. Because of this lock CPU-bound code will see 
	no gain in performance when using the Threading library\ldots
	\cite{quantstart}
\end{quote}
Instead, multi{\em processing} was used to create several Python instances
when necessary, each with their own lock that does not interfere with the
other processes. Unfortunately, this meant a large amount of code needed
to be restructured to ensure that all functions and the data passed to them
could be pickled. Implementation was done using the |Pool| class and the
|Process| class from Python's |multiprocessing| module.



% Mention memory speed bottleneck



\subsection{Precomputing All Distances}\label{ssec:precomputing}

