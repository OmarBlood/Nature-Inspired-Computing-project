%% BREAK LINES EVERY 80 CHARACTERS TO HELP GIT WITH MERGING
Results gathered from execution of the algorithm with the Western Sahara
dataset demonstrate the developed algorithm's efficacy in solving
the TSP. One improvement that the results from the Western Sahara
dataset indicate is a need for more effective local minima escaping 
techniques. Table~\ref{tab:last-gen-tiny} shows that in some cases
a high number of generations was required to achieve a near-optimal 
fitness, however Figure~\ref{fig:tiny-fitnesses} demonstrates a severe
plateau of best generational fitness around $500$ generations.

A more effective technique to escape local minima would reduce the 
number of generations with no improvement while still striving for 
near-optimal solutions.

While not as optimal, the results from the Uruguay dataset
do make sense and provide more useful information of where improvements
to the algorithm could be made. Extrapolation from 
Figure~\ref{fig:med-fitnesses} indicates that running more generations would likely 
result in a plateau of the best generational fitness much nearer the true
optimal solution. However, as execution for all datasets was done with
a constant population and mating pool size while the number of 
possible individuals grew at a rate of $n!$ for $n$ cities, future 
tests should be done with population and mating pool sizes that are not 
kept constant, but rather increase with the tour length. This would allow 
more variety in the population and hence could promote more diverse 
offspring, capable of escaping local minima faster and without 
reliance on severe mutation for randomness.